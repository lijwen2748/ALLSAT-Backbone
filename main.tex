\typeout{IJCAI--PRICAI--20 Instructions for Authors}
\documentclass{article}
\pdfpagewidth=8.5in
\pdfpageheight=11in
\usepackage{ijcai20}

\usepackage{times}
\usepackage{soul}
\usepackage{url}
\usepackage[hidelinks]{hyperref}
\usepackage[utf8]{inputenc}
\usepackage[small]{caption}
\usepackage{graphicx}
\usepackage{amsmath}
\usepackage{amsthm}
\usepackage{booktabs}
\usepackage{algorithm}
\usepackage{algorithmic}
\urlstyle{same}

\usepackage[table]{xcolor}
\usepackage{graphicx}
\usepackage{amssymb,amsmath,amsthm,amsfonts}
\usepackage[vlined,ruled,linesnumbered,algo2e]{algorithm2e}
\usepackage{color}

\newcommand{\tool}{{\sc BASolver}\xspace}
\newcommand{\ctool}{{\sc MBlocking}\xspace}
\newcommand{\bc}{{\sc BC}\xspace}
\newcommand{\nbc}{{\sc NBC}\xspace}
\newcommand{\bdd}{{\sc BDD}\xspace}

\newcommand{\BL}{{\sc BL}}



\newtheorem{theorem}{Theorem}
\newtheorem{lemma}{Lemma}
\newtheorem{corollary}{Corollary}
\newtheorem{proposition}{Proposition}
\newtheorem{definition}{Definition}
\newtheorem{example}{Example}
\newtheorem{remark}{Remark}

\begin{document}
\title{On Computing ALLSAT Using Backbone Information}
\maketitle
\begin{abstract}
The ALLSAT (All-Solutions) problems focus on finding every existing satisfiable assignment of a given propositional satisfiability formula. There are several different tasks require the finding of all satisfiable assignments, including model checking, automaton translation and test cases generation.
In this paper, we introduce \tool, a backbone based ALLSAT solver for propositional formulas. Comparing to the existing work, \tool uses backbone information to get shorter partial satisfiable assignments (partial solutions). More solutions are represented in a shorter partial solution, thus the efficiency of ALLSAT computing has been increased as less SAT solving is needed.
Experiments show that although finding backbone variables consumes additional computing time , but \tool is still quicker than existing tools,. including \ctool, \bc, \nbc and \bdd. Within the time limitation, the number of formulas which ALLSAT problem that \tool can solve (finish finding all satisfiable assignments of the formulas) is the most. For formulas that are solved by all of the 4 tools, \tool uses xx less computing time than \ctool, xx less time than \bc, xx less time than \nbc and xx less time than \bdd.
\end{abstract}
\section{Introduction}
SAT (satisfiable) problems have gained considerable attentions because the power of modern SAT solvers. SAT solvers and SAT problems are widely used in model checking~\cite{bmc}~\cite{ic3}, program analysis~\cite{klee}~\cite{cpachecker}~\cite{cbmc}, network verification ~\cite{lopes2015checking}~\cite{majumdar2014kuai}~\cite{zhang2012verification}, quantifier elimination~\cite{brauer2011existential}, and predicate abstraction~\cite{lahiri2003symbolic}.
In several of the areas that SAT solvers are applied, multiple satisfiable assignments are needed, in some of the applications including unbounded hardware model checking ~\cite{car}, logic minimization ~\cite{sapra2003sat}, and temporal logic planning ~\cite{aalta}, finding every satisfiable assignment is required. This paper focus on these kind of problems (ALLSAT problems) and proposed a solver \tool to find every satisfiable assignment of a given propositional formula efficiently.

Although finding a satisfiable assignment of a propositional formula is known as a NP problem, modern SAT solvers are still able to find the satisfiable assignment within an acceptable period of time. But for some of the formulas, directly enumerating every satisfiable assignment of the formula is infeasible due to the exponential number of satisfiable assignments. Therefore, finding all satisfiable assignments of the given formula remains challenging.
To overcome these challenges, there are mainly two kinds of strategies, blocking based strategies~\cite{mcmillan2002applying}, and non-blocking strategies ~\cite{grumberg2004memory}. Blocking based strategies uses SAT solver to find a solution, and adds the blocking clauses generated from the solution back to the original formula in order to avoid finding the same solution again. Computing the partial solution is a common technique used in blocking based strategies. For some of the cases that the SAT solver is efficient to enumerate every partial solution while fails to enumerate every individual full solution efficiently. 
Non-blocking strategies backtracks with the decision tree to find every satisfiable assignment of the given formula. Whenever a solution if found by the SAT solver, non-blocking strategies backtrack to a previous decision level based on the some strategies and change the assignment of the decision variable at that level to generate a new solution. When every branch in the decision tree is visited, ALLSAT computing of the formula is finished. 
Comparing to blocking based strategies, non-blocking strategies avoid the changes in the formula and use the complete information of the decision trees in the SAT solvers, while the idea of the blocking based strategies are more straightforward, and the implementations of blocking based strategies do not require changes to SAT solvers.

The ALLSAT solver \tool we introduced is a blocking based solver, we use Minisat v2.1.1~\cite{minisat} as the underlying SAT solver and implemented \tool with C++. The main innovation of \tool is that it uses the backbone information to reduce the length of the partial solutions. Every backbone variable is removed from the partial solution and added to the formula as a unit clause. In this way, \tool gets much shorter partial solutions without changing the correctness of the partial solutions. 
The number of SAT solving in the finding of every solution in \tool is reduced since every single shorter partial solution represents more full solutions. Comparing to the blocking based ALLSAT solvers with different coverage techniques~\cite{jin2005efficient}~\cite{morgado2005good}, \tool generates much shorter partial solutions for most of the formulas. Although finding backbone variables requires additional SAT solving, \tool is still faster than the state-of-the-art tool~\cite{ctool} (\ctool).

We compared \tool with 4 other tools, including 2 blocking based tools, \ctool~\cite{ctool} and \bc, one non-blocking based tool , \nbc and one BDD-based tool, \bdd~\cite{ietool}. Among the 5 tools, \tool solves (finds every solution of the given formulas) the most formulas (79) within the given time and memory limitation, while the number of solved formulas for \ctool, \bc, \nbc and \bdd  are 65, 64, 53 and 50 respectively. 
For the formulas that are solved by both \tool and \ctool, \tool uses 23\% less computing time than \ctool does.
\tool uses 37\%, 68\% and 31\% less computing time than \bc, \nbc and \bdd does respectively for the formulas that are mutually solved by \tool and one of the comparing tools.
\tool also gets the shortest blocking clauses among the three blocking based tools, the average length of blocking clauses for \tool is 1026, which is 20 times less than \ctool (22182) and 84\% less than \bc (6180). 

The remainder of the paper is organized as follows. We describe notations and preliminaries in Section \ref{sec:prel}. The algorithms of \tool are discussed in Section \ref{sec:meth}, experimental results are shown in Section \ref{sec:expr} and related work are discussed in Section \ref{sec:rela}. We conclude the paper in Section \ref{sec:conc}.


\section{Preliminaries} \label{sec:prel}
Notations
SAT formulas, variables, literals, clauses
unique satisfiable reason of and unique satisfied clause

Let $X$ be a finite set of Boolean variables. For every Boolean variable $x\in X$, $x$ can only be assigned to 0 or 1.
A literal $l$ is either a Boolean variable $x$ or its negation $\neg x$. For a literal $l$, the corresponding variable of $l$ is $x(l)$.
A clause $c$ is a disjunction of literals, a literal $l$ is in a clause $c$ ($l\in c$) if and only if $l$ appears in the disjunction that consist the clause $c$. A variable $x$ is in a clause $c$ if and only if literal $x$ or literal $\neg x$ appears in the disjunction that consist the clause $c$. 
A SAT formula $F$ is a conjunction of clauses, a clause $c$ is in the formula $F$ ($c\in F$) if and only if $c$ appears in the conjunction that consist the formula $F$. A variable $x$ is in a formula $F$ if and only if there exists a clause $c$ such that $x\in c$ and $c\in F$ or $\neg x\in c$ and $c\in F$. 

An \emph{assignment} $a$ of a given SAT formula $F$ is a function that maps each variable $x\in F$ to 0, 1, or $NOT_KNOWN$. 
For example, an assignment $a: \{x_1, x_2, ..., x_k\} \mapsto \{1, 1, ..., 0\}$ is an assignment of the formula $F$, where $k$ is the number of Boolean variables in $F$. 
For a variable $x\in F$, the value of $x$ in the assignment $a$ is $a(x)$, and the value of $F$ in the assignment $a$ is $a(F)$.
$\neg a(x)=1$ if $a(x)=0$ and $\neg a(x)=0$ if $a(x)=1$.
For an assignment $a$ if there does not exist a variable $x\in F$, such that $a(x)=NOT_KNOWN$, then $a$ is a full assignment of the given formula $F$. 
An assignment $p$ is a partial assignment if there exists at least one variable $x$ such that $p(x)=NOT_KNOWN$.

For a given assignment $v$, $v$ is a \emph{model (solution)} of the given formula $F$ if and only if the value of $F$ in the assignment $v$ is 1, i.e., $v\models F$ if and only if $v(F)=1$. If $v$ is a full assignment of $F$ then $v$ is also a full model (solution) of $F$. $v$ is a partial model (solution) of $F$ if $v$ is a partial assignment.
The models are written as the conjunctions of literals for short, for example, for a model $v$ if $v(a)=1$ and $v(b)=0$, then $v$ is written as $v=a\wedge \neg b$.

For a given partial model $v$, $v$ is able to represent at least two different full models $v_1$ and $v_2$, for every variable $x$ such that $v(x)\neq NOT_KNOWN$, $v_1(x)=v_2(x)=v(x)$, and for every variable $x'$ such that $v(x')=NOT_KNOWN$, $v_1(x')\neq v_2(x')$ and $v_1(x')\neq NOT_KNOWN$, $v_2(x')\neq NOT_KNOWN$.

For example, given a formula $F=(a\vee b)\wedge(a\vee \neg b)$, $v=a$ is partial model of $F$ and $v_1=a\wedge b$, $v_2=a\wedge \neg b$ are two different full models represents by $v$.


\begin{definition}[Backbone Variable]
For a given satisfiable formula $F$, and a variable $x\in F$, $x$ is a backbone variable if $x\wedge F$ or $\neg x \wedge F$ is unsatisfiable.
\end{definition}

\begin{lemma}[Backbone Assignment]
For a backbone variable $x$ of a given formula $F$, the value of $x$ must be always assigned to 1 or 0 in all models.
\end{lemma}

\begin{definition}[Essential Variables]
For a given formula $F$, a variable $x\in F$ and a full model $v\models F$, $x$ is an essential variable of $F$ with $v$, if and only if there exists a clause $c\in F$ such that $v(x)=1, x\in c$ and for every variable $x'\in c, x\neq x'$, $v(x')=0$. 
\end{definition}

An essential variable $x$ is the only reason that make a clause $c$ satisfiable in a formula $F$. For an essential variable $x$ with model $v$, the assignment $a$ is not a model of $F$ such that for every variable $x' \neq x $, $a(x')=v(x')$ and $a(x)=\neg v(x)$.

\section{ALLSAT Solvers} \label{sec:meth}
\subsection{Overview of ALLSAT solvers}
There are mainly 2 kinds of ALLSAT solvers, blocking based solvers and non-blocking based solvers.
Blocking based solvers use blocking clause to avoid finding the same solution. After finding a solution $v$, a partial solution $p$ that includes $v$ is computed and a blocking clause $c$ is generated by flipping the assignments of variables in $p$. Blocking clause $c$ is added to the original formula to block the solution $v$. The advantage of blocking clause is that more than one solution may be blocked by $c$ if there is more than 1 solution represented by $p$. But, the decision tree of the formula are updated every time a new blocking clause is added to the formula, and the decision information in the previous SAT solving are not fully usable in the following SAT soling.

Non-blocking based solvers use backtrack techniques to find all solutions of the given formula. A solution $v$ is first computed, then non-blocking solvers backtrack to a previous decision level based on the backjumping stategies and change the assignment of the decision variable to get a new solution $v'$. The complete decision information are always usable for the following SAT solving as no blocking clause is added to the original formula. But, non-blocking solvers can not take the advantage of partial solutions and state explosion is a servious problem when there are many variabls and clauses in the formula.
\subsection{Overview of Algorithms}
Figure \ref{} shows the overview of \tool. For a given satisfiable formula $F$, 

\subsection{Computing Backbone Variables}

\subsection{Computing Partial Assignments using Backbone Information}

\subsection{Computing Partial Assignment using Essential Variables}

\subsection{Coverage for Clauses}



\section{Evaluation} \label{sec:expr}
minimal blocking experiments
in 1 hour and in 10 hours
different benchmarks
different solvers (bc, nbc, bdd, allsat, tool)

the length of partial assignments 

for formulas that does not have backbone variables, due to different covering strategy, the performance are different


\section{Related Work} \label{sec:rela}
1. naive blocking
2. coverage blocking
3. minimal blocking
4. BDD (non blocking)
6. P systems, SAT and ALLSAT

\section{Conclusion} \label{sec:conc}
shorter partial assignments


\end{document}
