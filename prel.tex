\section{Preliminaries} \label{sec:prel}
Let $X$ be a finite set of Boolean variables. For every Boolean variable $x\in X$, $x$ can only be assigned to 0 or 1, otherwise the value of $x$ is NOT\_KNOWN.
A literal $l$ is either a Boolean variable $x$ or its negation $\neg x$. For a literal $l$, the corresponding variable of $l$ is $x(l)$.
A clause $c$ is a disjunction of literals, a literal $l$ is in a clause $c$ ($l\in c$) if and only if $l$ appears in the disjunction that compose the clause $c$. A variable $x$ is in a clause $c$ if and only if literal $x$ or literal $\neg x$ is in the clause $c$.
A SAT formula $F$ is a conjunction of clauses, a clause $c$ is in the formula $F$ ($c\in F$) if and only if $c$ appears in the conjunction that compose the formula $F$. A variable $x$ is in a formula $F$ if and only if there exists a clause $c\in F$ such that $x\in c$ or $\neg x\in c$. 

An \emph{assignment} $a$ of a given SAT formula $F$ is a function that maps each variable $x\in F$ to 0, 1, or NOT\_KNOWN. 
For example, an assignment $a: \{x_1, x_2, ..., x_k\} \mapsto \{1, 1, ..., 0\}$ is an assignment of the formula $F$, where $k$ is the number of Boolean variables in $F$. 
For a variable $x\in F$, the value of $x$ in the assignment $a$ is $a(x)$, the value of a clause $c\in F$ in the assignment $a$ is $a(c)$, and the value of $F$ in the assignment $a$ is $a(F)$.
$\neg a(x)=1$ if $a(x)=0$ and $\neg a(x)=0$ if $a(x)=1$.
For an assignment $a$ if there does not exist a variable $x\in F$, such that $a(x)$=NOT\_KNOWN, then $a$ is a full assignment of the given formula $F$. 
An assignment $p$ is a partial assignment if there exists at least one variable $x$ such that $p(x)$=NOT\_KNOWN.

For a given assignment $v$, $v$ is a \emph{solution} of the given formula $F$ if and only if the value of $F$ in the assignment $v$ is 1, i.e., $v\models F$ if and only if $v(F)=1$. If $v$ is a full assignment of $F$ then $v$ is also a full solution of $F$. $v$ is a partial solution of $F$ if $v$ is a partial assignment.
The solutions are written as the conjunctions of literals for short, for example, for a solution $v$ such that $v(a)=1$ and $v(b)=0$, then $v$ is written as $v=a\wedge \neg b$.

For a given partial solution $v$, $v$ is able to represent at least two different full solutions $v_1$ and $v_2$, for every variable $x$ such that $v(x) \neq $NOT\_KNOWN, $v_1(x) = v_2(x) = v(x)$, and for every variable $x'$ such that $v(x')$ = NOT\_KNOWN, $v_1(x') \neq v_2(x')$ and $v_1(x') \neq $ NOT\_KNOWN, $v_2(x') \neq $NOT\_KNOWN.

For example, given a formula $F=(a\vee b)\wedge(a\vee \neg b)$, $v=a$ is partial solution of $F$ and $v_1=a\wedge b$, $v_2=a\wedge \neg b$ are two different full solution represents by $v$.


\begin{definition}[Backbone Variable]
For a given satisfiable formula $F$, and a variable $x\in F$, $x$ is a backbone variable if $x\wedge F$ or $\neg x \wedge F$ is unsatisfiable.
\end{definition}

\begin{lemma}[Backbone Assignment]
For a backbone variable $x$ of a given formula $F$, the value of $x$ must be always assigned to 1 or 0 in all solutions.
\end{lemma}

\begin{definition}[Essential Literals]
For a given formula $F$, a literal $x\in F$ and a full solution $v\models F$, $x$ is an essential literal of $F$ with $v$, if and only if there exists a clause $c\in F$ such that $v(x)=1, x\in c$ and for every other literal $x'\in c, x\neq x'$, $v(x')=0$. 
\end{definition}

An essential variable $x$ is the only reason that make a clause $c$ satisfiable in a formula $F$ with the given solution. For an essential variable $x$ with solution $v$, the assignment $a$ is not a solution of $F$ such that for every variable $x' \neq x $, $a(x')=v(x')$ and $a(x)=\neg v(x)$.

For example, given a formula $F=(a\vee b)\wedge(a \vee \neg b)$, $a$ is a backbone variable since $\neg a \wedge F$ is unsatisfiable.
There are only two solutions of $F$, $v_1=a\wedge b$, and $v_2=a\wedge \neg b$, the assignment of backbone variable $a$ is always 1 in every solution of $F$. 
For the solution $v_1$, the literal $a$ is an essential literal since for the clause $a\vee \neg b$, $v_1(a)=1$ and for every other literal (the literal $\neg b$) in $a\vee \neg b$, $v_1(\neg b)=0$.