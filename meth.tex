\section{ALLSAT Solvers} \label{sec:meth}
\subsection{Overview of ALLSAT solvers}
There are mainly 2 kinds of ALLSAT solvers, blocking based solvers and non-blocking based solvers.
Blocking based solvers use blocking clause to avoid finding the same solution. After finding a solution $v$, a partial solution $p$ that includes $v$ is computed and a blocking clause $c$ is generated by flipping the assignments of variables in $p$. Blocking clause $c$ is added to the original formula to block the solution $v$. The advantage of blocking clause is that more than one solution may be blocked by $c$ if there is more than 1 solution represented by $p$. But, the decision tree of the formula are updated every time a new blocking clause is added to the formula, and the decision information in the previous SAT solving are not fully usable in the following SAT soling.

Non-blocking based solvers use backtrack techniques to find all solutions of the given formula. A solution $v$ is first computed, then non-blocking solvers backtrack to a previous decision level based on the backjumping stategies and change the assignment of the decision variable to get a new solution $v'$. The complete decision information are always usable for the following SAT solving as no blocking clause is added to the original formula. But, non-blocking solvers can not take the advantage of partial solutions and state explosion is a servious problem when there are many variabls and clauses in the formula.
\subsection{Overview of Algorithms}
Figure \ref{} shows the overview of \tool. For a given satisfiable formula $F$, 

\subsection{Computing Backbone Variables}

\subsection{Computing Partial Assignments using Backbone Information}

\subsection{Computing Partial Assignment using Essential Variables}

\subsection{Coverage for Clauses}


